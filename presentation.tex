% Options for packages loaded elsewhere
% Options for packages loaded elsewhere
\PassOptionsToPackage{unicode}{hyperref}
\PassOptionsToPackage{hyphens}{url}
\PassOptionsToPackage{dvipsnames,svgnames,x11names}{xcolor}
%
\documentclass[
]{beamer}
\usepackage{xcolor}
\usepackage{amsmath,amssymb}
\setcounter{secnumdepth}{-\maxdimen} % remove section numbering
\usepackage{iftex}
\ifPDFTeX
  \usepackage[T1]{fontenc}
  \usepackage[utf8]{inputenc}
  \usepackage{textcomp} % provide euro and other symbols
\else % if luatex or xetex
  \usepackage{unicode-math} % this also loads fontspec
  \defaultfontfeatures{Scale=MatchLowercase}
  \defaultfontfeatures[\rmfamily]{Ligatures=TeX,Scale=1}
\fi
\usepackage{lmodern}
\ifPDFTeX\else
  % xetex/luatex font selection
\fi
% Use upquote if available, for straight quotes in verbatim environments
\IfFileExists{upquote.sty}{\usepackage{upquote}}{}
\IfFileExists{microtype.sty}{% use microtype if available
  \usepackage[]{microtype}
  \UseMicrotypeSet[protrusion]{basicmath} % disable protrusion for tt fonts
}{}
\makeatletter
\@ifundefined{KOMAClassName}{% if non-KOMA class
  \IfFileExists{parskip.sty}{%
    \usepackage{parskip}
  }{% else
    \setlength{\parindent}{0pt}
    \setlength{\parskip}{6pt plus 2pt minus 1pt}}
}{% if KOMA class
  \KOMAoptions{parskip=half}}
\makeatother
% Make \paragraph and \subparagraph free-standing
\makeatletter
\ifx\paragraph\undefined\else
  \let\oldparagraph\paragraph
  \renewcommand{\paragraph}{
    \@ifstar
      \xxxParagraphStar
      \xxxParagraphNoStar
  }
  \newcommand{\xxxParagraphStar}[1]{\oldparagraph*{#1}\mbox{}}
  \newcommand{\xxxParagraphNoStar}[1]{\oldparagraph{#1}\mbox{}}
\fi
\ifx\subparagraph\undefined\else
  \let\oldsubparagraph\subparagraph
  \renewcommand{\subparagraph}{
    \@ifstar
      \xxxSubParagraphStar
      \xxxSubParagraphNoStar
  }
  \newcommand{\xxxSubParagraphStar}[1]{\oldsubparagraph*{#1}\mbox{}}
  \newcommand{\xxxSubParagraphNoStar}[1]{\oldsubparagraph{#1}\mbox{}}
\fi
\makeatother


\usepackage{longtable,booktabs,array}
\usepackage{calc} % for calculating minipage widths
% Correct order of tables after \paragraph or \subparagraph
\usepackage{etoolbox}
\makeatletter
\patchcmd\longtable{\par}{\if@noskipsec\mbox{}\fi\par}{}{}
\makeatother
% Allow footnotes in longtable head/foot
\IfFileExists{footnotehyper.sty}{\usepackage{footnotehyper}}{\usepackage{footnote}}
\makesavenoteenv{longtable}
\usepackage{graphicx}
\makeatletter
\newsavebox\pandoc@box
\newcommand*\pandocbounded[1]{% scales image to fit in text height/width
  \sbox\pandoc@box{#1}%
  \Gscale@div\@tempa{\textheight}{\dimexpr\ht\pandoc@box+\dp\pandoc@box\relax}%
  \Gscale@div\@tempb{\linewidth}{\wd\pandoc@box}%
  \ifdim\@tempb\p@<\@tempa\p@\let\@tempa\@tempb\fi% select the smaller of both
  \ifdim\@tempa\p@<\p@\scalebox{\@tempa}{\usebox\pandoc@box}%
  \else\usebox{\pandoc@box}%
  \fi%
}
% Set default figure placement to htbp
\def\fps@figure{htbp}
\makeatother





\setlength{\emergencystretch}{3em} % prevent overfull lines

\providecommand{\tightlist}{%
  \setlength{\itemsep}{0pt}\setlength{\parskip}{0pt}}



 


\makeatletter
\@ifpackageloaded{caption}{}{\usepackage{caption}}
\AtBeginDocument{%
\ifdefined\contentsname
  \renewcommand*\contentsname{Table of contents}
\else
  \newcommand\contentsname{Table of contents}
\fi
\ifdefined\listfigurename
  \renewcommand*\listfigurename{List of Figures}
\else
  \newcommand\listfigurename{List of Figures}
\fi
\ifdefined\listtablename
  \renewcommand*\listtablename{List of Tables}
\else
  \newcommand\listtablename{List of Tables}
\fi
\ifdefined\figurename
  \renewcommand*\figurename{Figure}
\else
  \newcommand\figurename{Figure}
\fi
\ifdefined\tablename
  \renewcommand*\tablename{Table}
\else
  \newcommand\tablename{Table}
\fi
}
\@ifpackageloaded{float}{}{\usepackage{float}}
\floatstyle{ruled}
\@ifundefined{c@chapter}{\newfloat{codelisting}{h}{lop}}{\newfloat{codelisting}{h}{lop}[chapter]}
\floatname{codelisting}{Listing}
\newcommand*\listoflistings{\listof{codelisting}{List of Listings}}
\makeatother
\makeatletter
\makeatother
\makeatletter
\@ifpackageloaded{caption}{}{\usepackage{caption}}
\@ifpackageloaded{subcaption}{}{\usepackage{subcaption}}
\makeatother
\usepackage{bookmark}
\IfFileExists{xurl.sty}{\usepackage{xurl}}{} % add URL line breaks if available
\urlstyle{same}
\hypersetup{
  pdftitle={Analyse factorielle des correspondances des causes de décès par pays},
  pdfauthor={Mkrtchyan, Touré},
  colorlinks=true,
  linkcolor={blue},
  filecolor={Maroon},
  citecolor={Blue},
  urlcolor={Blue},
  pdfcreator={LaTeX via pandoc}}


\title{Analyse factorielle des correspondances des causes de décès par
pays}
\author{Mkrtchyan, Touré}
\date{2026-01-01}
\begin{document}
\maketitle


\begin{center}\rule{0.5\linewidth}{0.5pt}\end{center}

\subsection{Analyse exploratoire}\label{analyse-exploratoire}

\begin{itemize}
\tightlist
\item
  Distribution des causes par pays
\item
  Comparaison visuelle des profils
\item
  Mise en évidence de disparités importantes
\end{itemize}

\begin{center}\rule{0.5\linewidth}{0.5pt}\end{center}

\subsection{Visualisation -- Arménie}\label{visualisation-armuxe9nie}

```r

\begin{center}\rule{0.5\linewidth}{0.5pt}\end{center}

\subsection{Comparaison des pays}\label{comparaison-des-pays}

\begin{itemize}
\tightlist
\item
  Profils de mortalité très contrastés
\item
  Pays à faible revenu vs pays développés
\item
  Importance variable des maladies chroniques
\end{itemize}

\begin{center}\rule{0.5\linewidth}{0.5pt}\end{center}

\subsection{Profils ligne et colonne}\label{profils-ligne-et-colonne}

\begin{itemize}
\tightlist
\item
  Profil ligne : distribution des causes pour un pays
\item
  Profil colonne : répartition des pays pour une cause
\item
  Base de l'AFC
\end{itemize}

\begin{center}\rule{0.5\linewidth}{0.5pt}\end{center}

\subsection{Profils avec ggplot2}\label{profils-avec-ggplot2}

```r

\begin{center}\rule{0.5\linewidth}{0.5pt}\end{center}

\subsection{Test du khi-deux}\label{test-du-khi-deux}

\begin{itemize}
\tightlist
\item
  Hypothèses :

  \begin{itemize}
  \tightlist
  \item
    H₀ : indépendance pays / causes
  \item
    H₁ : dépendance
  \end{itemize}
\item
  Statistique du khi-deux très élevée
\item
  p-value ≈ 0
\end{itemize}

👉 \textbf{Rejet clair de l'hypothèse d'indépendance}

\begin{center}\rule{0.5\linewidth}{0.5pt}\end{center}

\subsection{V de Cramer}\label{v-de-cramer}

\begin{itemize}
\tightlist
\item
  Mesure de l'intensité de la dépendance
\item
  Résultat obtenu :\\
  \textbf{V ≈ 0.15}
\item
  Dépendance \textbf{faible à modérée}
\end{itemize}

👉 Résultat cohérent avec la taille de l'échantillon

\begin{center}\rule{0.5\linewidth}{0.5pt}\end{center}

\subsection{Lancement de l'AFC}\label{lancement-de-lafc}

```r

\begin{center}\rule{0.5\linewidth}{0.5pt}\end{center}

\subsection{Critère du bâton
brisé}\label{crituxe8re-du-buxe2ton-brisuxe9}

\begin{itemize}
\tightlist
\item
  Sélection du nombre d'axes pertinents
\item
  Comparaison inertie réelle / inertie théorique
\item
  Conservation des premiers axes explicatifs
\end{itemize}

\begin{center}\rule{0.5\linewidth}{0.5pt}\end{center}

\subsection{Plan factoriel (axes 1 et
2)}\label{plan-factoriel-axes-1-et-2}

```r

\begin{center}\rule{0.5\linewidth}{0.5pt}\end{center}

\subsection{Contributions des lignes
(pays)}\label{contributions-des-lignes-pays}

\begin{itemize}
\tightlist
\item
  Pays les plus contributifs aux axes
\item
  Interprétation géographique et sanitaire
\item
  Opposition pays développés / pays en développement
\end{itemize}

\begin{center}\rule{0.5\linewidth}{0.5pt}\end{center}

\subsection{Contributions des colonnes
(causes)}\label{contributions-des-colonnes-causes}

\begin{itemize}
\tightlist
\item
  Causes dominantes sur chaque axe
\item
  Axes interprétables en termes épidémiologiques
\item
  Lecture conjointe pays--causes
\end{itemize}

\begin{center}\rule{0.5\linewidth}{0.5pt}\end{center}

\subsection{Distances au centre de
gravité}\label{distances-au-centre-de-gravituxe9}

\begin{itemize}
\tightlist
\item
  Distance = atypie du profil
\item
  Certains pays très éloignés
\item
  Indicateur de spécificité sanitaire
\end{itemize}

\begin{center}\rule{0.5\linewidth}{0.5pt}\end{center}

\subsection{Conclusion}\label{conclusion}

\begin{itemize}
\tightlist
\item
  Forte dépendance statistique pays / causes
\item
  Intensité modérée (V de Cramer)
\item
  AFC très pertinente pour l'interprétation
\item
  Résultats cohérents avec les contextes sanitaires
\end{itemize}

\begin{center}\rule{0.5\linewidth}{0.5pt}\end{center}

\subsection{Références}\label{ruxe9fuxe9rences}

\begin{itemize}
\tightlist
\item
  Cours de statistique multidimensionnelle
\item
  Documentation R (\texttt{FactoMineR}, \texttt{ggplot2})
\item
  Sources de données internationales (OMS, etc.)
\end{itemize}




\end{document}
