% Options for packages loaded elsewhere
% Options for packages loaded elsewhere
\PassOptionsToPackage{unicode}{hyperref}
\PassOptionsToPackage{hyphens}{url}
\PassOptionsToPackage{dvipsnames,svgnames,x11names}{xcolor}
%
\documentclass[
]{beamer}
\usepackage{xcolor}
\usepackage{amsmath,amssymb}
\setcounter{secnumdepth}{5}
\usepackage{iftex}
\ifPDFTeX
  \usepackage[T1]{fontenc}
  \usepackage[utf8]{inputenc}
  \usepackage{textcomp} % provide euro and other symbols
\else % if luatex or xetex
  \usepackage{unicode-math} % this also loads fontspec
  \defaultfontfeatures{Scale=MatchLowercase}
  \defaultfontfeatures[\rmfamily]{Ligatures=TeX,Scale=1}
\fi
\usepackage{lmodern}
\ifPDFTeX\else
  % xetex/luatex font selection
\fi
% Use upquote if available, for straight quotes in verbatim environments
\IfFileExists{upquote.sty}{\usepackage{upquote}}{}
\IfFileExists{microtype.sty}{% use microtype if available
  \usepackage[]{microtype}
  \UseMicrotypeSet[protrusion]{basicmath} % disable protrusion for tt fonts
}{}
\makeatletter
\@ifundefined{KOMAClassName}{% if non-KOMA class
  \IfFileExists{parskip.sty}{%
    \usepackage{parskip}
  }{% else
    \setlength{\parindent}{0pt}
    \setlength{\parskip}{6pt plus 2pt minus 1pt}}
}{% if KOMA class
  \KOMAoptions{parskip=half}}
\makeatother
% Make \paragraph and \subparagraph free-standing
\makeatletter
\ifx\paragraph\undefined\else
  \let\oldparagraph\paragraph
  \renewcommand{\paragraph}{
    \@ifstar
      \xxxParagraphStar
      \xxxParagraphNoStar
  }
  \newcommand{\xxxParagraphStar}[1]{\oldparagraph*{#1}\mbox{}}
  \newcommand{\xxxParagraphNoStar}[1]{\oldparagraph{#1}\mbox{}}
\fi
\ifx\subparagraph\undefined\else
  \let\oldsubparagraph\subparagraph
  \renewcommand{\subparagraph}{
    \@ifstar
      \xxxSubParagraphStar
      \xxxSubParagraphNoStar
  }
  \newcommand{\xxxSubParagraphStar}[1]{\oldsubparagraph*{#1}\mbox{}}
  \newcommand{\xxxSubParagraphNoStar}[1]{\oldsubparagraph{#1}\mbox{}}
\fi
\makeatother


\usepackage{longtable,booktabs,array}
\usepackage{calc} % for calculating minipage widths
% Correct order of tables after \paragraph or \subparagraph
\usepackage{etoolbox}
\makeatletter
\patchcmd\longtable{\par}{\if@noskipsec\mbox{}\fi\par}{}{}
\makeatother
% Allow footnotes in longtable head/foot
\IfFileExists{footnotehyper.sty}{\usepackage{footnotehyper}}{\usepackage{footnote}}
\makesavenoteenv{longtable}
\usepackage{graphicx}
\makeatletter
\newsavebox\pandoc@box
\newcommand*\pandocbounded[1]{% scales image to fit in text height/width
  \sbox\pandoc@box{#1}%
  \Gscale@div\@tempa{\textheight}{\dimexpr\ht\pandoc@box+\dp\pandoc@box\relax}%
  \Gscale@div\@tempb{\linewidth}{\wd\pandoc@box}%
  \ifdim\@tempb\p@<\@tempa\p@\let\@tempa\@tempb\fi% select the smaller of both
  \ifdim\@tempa\p@<\p@\scalebox{\@tempa}{\usebox\pandoc@box}%
  \else\usebox{\pandoc@box}%
  \fi%
}
% Set default figure placement to htbp
\def\fps@figure{htbp}
\makeatother





\setlength{\emergencystretch}{3em} % prevent overfull lines

\providecommand{\tightlist}{%
  \setlength{\itemsep}{0pt}\setlength{\parskip}{0pt}}



 


\makeatletter
\@ifpackageloaded{caption}{}{\usepackage{caption}}
\AtBeginDocument{%
\ifdefined\contentsname
  \renewcommand*\contentsname{Table of contents}
\else
  \newcommand\contentsname{Table of contents}
\fi
\ifdefined\listfigurename
  \renewcommand*\listfigurename{List of Figures}
\else
  \newcommand\listfigurename{List of Figures}
\fi
\ifdefined\listtablename
  \renewcommand*\listtablename{List of Tables}
\else
  \newcommand\listtablename{List of Tables}
\fi
\ifdefined\figurename
  \renewcommand*\figurename{Figure}
\else
  \newcommand\figurename{Figure}
\fi
\ifdefined\tablename
  \renewcommand*\tablename{Table}
\else
  \newcommand\tablename{Table}
\fi
}
\@ifpackageloaded{float}{}{\usepackage{float}}
\floatstyle{ruled}
\@ifundefined{c@chapter}{\newfloat{codelisting}{h}{lop}}{\newfloat{codelisting}{h}{lop}[chapter]}
\floatname{codelisting}{Listing}
\newcommand*\listoflistings{\listof{codelisting}{List of Listings}}
\makeatother
\makeatletter
\makeatother
\makeatletter
\@ifpackageloaded{caption}{}{\usepackage{caption}}
\@ifpackageloaded{subcaption}{}{\usepackage{subcaption}}
\makeatother
\usepackage{bookmark}
\IfFileExists{xurl.sty}{\usepackage{xurl}}{} % add URL line breaks if available
\urlstyle{same}
\hypersetup{
  pdftitle={Analyse factorielle des correspondances des causes de décès par pays},
  pdfauthor={MKRTCHYAN Yester, TOURE Fatoumata Salihou},
  colorlinks=true,
  linkcolor={blue},
  filecolor={Maroon},
  citecolor={Blue},
  urlcolor={Blue},
  pdfcreator={LaTeX via pandoc}}


\title{Analyse factorielle des correspondances des causes de décès par
pays}
\author{MKRTCHYAN Yester, TOURE Fatoumata Salihou}
\date{}
\begin{document}
\maketitle

\renewcommand*\contentsname{Table of contents}
{
\hypersetup{linkcolor=}
\setcounter{tocdepth}{2}
\tableofcontents
}

\section{Introduction}\label{introduction}

En 2019, en Afrique subsaharienne, les maladies non transmissibles
représentent environ 35 à 40 \% des décès chez les adultes, d'après les
données mondiales de l'OMS. Intuitivement, les infections pourraient
pourtant être considérées comme dominantes, compte tenu du niveau de
développement des pays. Cela soulève plusieurs questions centrales pour
l'analyse : \emph{Existe-t-il un lien entre le niveau de développement
des pays et les différentes causes de décès ? Quels pays présentent des
profils similaires de mortalité ?} Afin d'y répondre rigoureusement, il
est pertinent de réaliser une Analyse Factorielle des Correspondances
(AFC) sur les données enregistrées.

\section{Analyse factorielle des correspondances des causes de décès par
pays}\label{analyse-factorielle-des-correspondances-des-causes-de-duxe9cuxe8s-par-pays}

\subsection{Préparation des
données}\label{pruxe9paration-des-donnuxe9es}

Pour cette analyse, nous avons utilisé un jeu de données disponible sur
Kaggle, portant sur les causes de mortalité dans le monde dont le lien
se trouve en annexe. Le jeu de données original couvre la période 1990 à
2019 et recense les effectifs de décès pour 204 pays et territoires,
organisés sous la forme d'un tableau pays × causes de décès.

Afin de concentrer l'étude sur l'année la plus récente, nous avons
extrait uniquement les observations de 2019. Une vérification préalable
a confirmé l'absence de valeurs manquantes dans les variables retenues.
Les données extraites ont ensuite été enregistrées dans un nouveau
fichier CSV, distinct du fichier initial.

La variable correspondant à l'année a été supprimée du fait que toutes
les lignes concernent 2019. De même, parmi les deux colonnes
descriptives des pays (nom complet et abréviation), seule la colonne des
abréviations a été conservée, les correspondances avec les noms complets
étant stockées dans un fichier séparé.

Pour améliorer la lisibilité et faciliter l'interprétation graphique
dans l'Analyse Factorielle des Correspondances (AFC), nous avons procédé
à un encodage des différentes causes de décès, chaque cause étant
associée à un identifiant plus court. Cet encodage a été enregistré dans
un document séparé, qui sera également fourni dans le rapport.

Le jeu de données final comporte ainsi 204 lignes correspondant aux
abréviations des pays et territoires, et des colonnes représentant les
différentes causes de décès encodées, avec pour valeurs les effectifs
observés. En étant deux variables qualitatives il convient de procéder à
une AFC.

\subsection{Analyse exploratoire}\label{analyse-exploratoire}

\subsubsection{Chargement des données}\label{chargement-des-donnuxe9es}

La première étape de l'analyse consiste à charger les librairies
nécessaires ainsi que le jeu de données à analyser dans le script R qui
est le fichier CSV nettoyé. Après l'importation, une première
visualisation globale du tableau est réalisée afin de vérifier la
structure des données et la cohérence des variables.

Au moment du chargement de données, la première colonne du jeu de
données est désigné comme étant la colonne d'identification des lignes
et elle est ensuite supprimée, afin de ne conserver que les causes de
décès sous forme numérique. Cette étape permet de faciliter les
traitements ultérieurs et l'accès direct aux informations par pays.

Le jeu de données obtenu est stocké dans la variable ``morts'' pour le
reste de l'analyse. Une vérification du jeu de données à l'aide de
l'affichage des premières lignes confirme la bonne manipulation des
données et l'absence d'anomalies apparentes.

\subsubsection{Visualisation des données par
barplots}\label{visualisation-des-donnuxe9es-par-barplots}

\pandocbounded{\includegraphics[keepaspectratio]{rapport_files/figure-pdf/unnamed-chunk-2-1.pdf}}

\pandocbounded{\includegraphics[keepaspectratio]{rapport_files/figure-pdf/unnamed-chunk-3-1.pdf}}

\pandocbounded{\includegraphics[keepaspectratio]{rapport_files/figure-pdf/unnamed-chunk-4-1.pdf}}

\subsubsection{Comparaison de pays}\label{comparaison-de-pays}

Les Figures 1 à 3 présentent des diagrammes en barres horizontales
illustrant le nombre de décès par cause respectivement en Arménie, au
Mali et en France. Ces représentations graphiques ont pour objectif de
comparer la structure de la mortalité selon les principales causes de
décès dans chacun des pays étudiés.

Les causes de décès sont représentées sur l'axe des ordonnées, tandis
que l'axe des abscisses indique le nombre de morts. Ce choix de
visualisation permet une lecture claire et facilite la comparaison entre
catégories.

L'analyse met en évidence des profils de mortalité contrastés. En
Arménie, en 2019, les maladies cardiovasculaires (C12) sont responsables
de plus de 12 000 décès, tandis que le néoplasme (C20) en représente
environ 6 000. En France, ces mêmes causes figurent également parmi les
principales causes de mortalité, avec plus de 15 000 décès pour le
néoplasme et environ 20 000 pour la cause cardiovasculaires. À
l'inverse, au Mali, les maladies transmissibles ainsi que certaines
maladies infectueuses occupent une place plus importante dans la
mortalité totale : les entraîné environ 32 000 décès, les maladies
cardiovasculaires (C12) près de 20 000 décès, et l'automutilation et la
malnutrition (C18 et C25) environ 14 000 décès chacune.

Cette analyse descriptive constitue une première étape exploratoire.
Elle permet de dégager des tendances générales et justifie la poursuite
de l'étude par des analyses comparatives plus approfondies entre pays et
groupes de causes de décès.

\subsubsection{Profils ligne et colonne}\label{profils-ligne-et-colonne}

Afin d'approfondir l'analyse descriptive, des profils lignes et des
profils colonnes ont été construits à partir du tableau de contingence
des décès par cause et par pays. Le profil ligne d'un pays correspond à
la répartition relative des causes de décès à l'intérieur de ce pays :
chaque valeur représente la proportion d'une cause donnée parmi
l'ensemble des décès du pays considéré. Par exemple, en Afghanistan la
meningite reperesente 0.7\% des causes des décès en 2019 (0.007 =
1563/(1563+1775+\ldots+485 +1940)). Ainsi, la somme des proportions sur
une ligne est égale à 1, ce qui permet de comparer les structures de
mortalité indépendamment du niveau absolu de mortalité.

\begin{longtable}[]{@{}
  >{\raggedright\arraybackslash}p{(\linewidth - 16\tabcolsep) * \real{0.0476}}
  >{\raggedleft\arraybackslash}p{(\linewidth - 16\tabcolsep) * \real{0.1190}}
  >{\raggedleft\arraybackslash}p{(\linewidth - 16\tabcolsep) * \real{0.1190}}
  >{\raggedleft\arraybackslash}p{(\linewidth - 16\tabcolsep) * \real{0.1190}}
  >{\raggedleft\arraybackslash}p{(\linewidth - 16\tabcolsep) * \real{0.1190}}
  >{\raggedleft\arraybackslash}p{(\linewidth - 16\tabcolsep) * \real{0.1190}}
  >{\raggedleft\arraybackslash}p{(\linewidth - 16\tabcolsep) * \real{0.1190}}
  >{\raggedleft\arraybackslash}p{(\linewidth - 16\tabcolsep) * \real{0.1190}}
  >{\raggedleft\arraybackslash}p{(\linewidth - 16\tabcolsep) * \real{0.1190}}@{}}
\caption{Profils lignes pour quelques pays et causes
sélectionnés}\tabularnewline
\toprule\noalign{}
\begin{minipage}[b]{\linewidth}\raggedright
\end{minipage} & \begin{minipage}[b]{\linewidth}\raggedleft
C1
\end{minipage} & \begin{minipage}[b]{\linewidth}\raggedleft
C2
\end{minipage} & \begin{minipage}[b]{\linewidth}\raggedleft
C3
\end{minipage} & \begin{minipage}[b]{\linewidth}\raggedleft
C4
\end{minipage} & \begin{minipage}[b]{\linewidth}\raggedleft
C5
\end{minipage} & \begin{minipage}[b]{\linewidth}\raggedleft
C6
\end{minipage} & \begin{minipage}[b]{\linewidth}\raggedleft
C7
\end{minipage} & \begin{minipage}[b]{\linewidth}\raggedleft
C8
\end{minipage} \\
\midrule\noalign{}
\endfirsthead
\toprule\noalign{}
\begin{minipage}[b]{\linewidth}\raggedright
\end{minipage} & \begin{minipage}[b]{\linewidth}\raggedleft
C1
\end{minipage} & \begin{minipage}[b]{\linewidth}\raggedleft
C2
\end{minipage} & \begin{minipage}[b]{\linewidth}\raggedleft
C3
\end{minipage} & \begin{minipage}[b]{\linewidth}\raggedleft
C4
\end{minipage} & \begin{minipage}[b]{\linewidth}\raggedleft
C5
\end{minipage} & \begin{minipage}[b]{\linewidth}\raggedleft
C6
\end{minipage} & \begin{minipage}[b]{\linewidth}\raggedleft
C7
\end{minipage} & \begin{minipage}[b]{\linewidth}\raggedleft
C8
\end{minipage} \\
\midrule\noalign{}
\endhead
\bottomrule\noalign{}
\endlastfoot
AFG & 0.0071816 & 0.0081556 & 0.0025730 & 0.0057158 & 0.0024352 &
0.0077513 & 0.0230425 & 0.0185535 \\
ARM & 0.0001788 & 0.0312187 & 0.0060435 & 0.0001430 & 0.0000000 &
0.0015735 & 0.0054713 & 0.0002861 \\
MLI & 0.0332092 & 0.0065039 & 0.0017241 & 0.0788586 & 0.0954897 &
0.0052678 & 0.0122810 & 0.0186735 \\
FRA & 0.0003979 & 0.0810118 & 0.0140813 & 0.0083357 & 0.0000000 &
0.0016622 & 0.0009473 & 0.0000810 \\
MOZ & 0.0109682 & 0.0061453 & 0.0013705 & 0.0141875 & 0.0825047 &
0.0028777 & 0.0072786 & 0.0083276 \\
AUS & 0.0002859 & 0.0698592 & 0.0149352 & 0.0010193 & 0.0000000 &
0.0012182 & 0.0020013 & 0.0000808 \\
ARG & 0.0013440 & 0.0357363 & 0.0090283 & 0.0039460 & 0.0000000 &
0.0016793 & 0.0085476 & 0.0011185 \\
USA & 0.0004042 & 0.0507657 & 0.0113620 & 0.0021482 & 0.0000000 &
0.0012751 & 0.0062466 & 0.0003492 \\
\end{longtable}

\begin{longtable}[]{@{}
  >{\raggedright\arraybackslash}p{(\linewidth - 16\tabcolsep) * \real{0.0476}}
  >{\raggedleft\arraybackslash}p{(\linewidth - 16\tabcolsep) * \real{0.1190}}
  >{\raggedleft\arraybackslash}p{(\linewidth - 16\tabcolsep) * \real{0.1190}}
  >{\raggedleft\arraybackslash}p{(\linewidth - 16\tabcolsep) * \real{0.1190}}
  >{\raggedleft\arraybackslash}p{(\linewidth - 16\tabcolsep) * \real{0.1190}}
  >{\raggedleft\arraybackslash}p{(\linewidth - 16\tabcolsep) * \real{0.1190}}
  >{\raggedleft\arraybackslash}p{(\linewidth - 16\tabcolsep) * \real{0.1190}}
  >{\raggedleft\arraybackslash}p{(\linewidth - 16\tabcolsep) * \real{0.1190}}
  >{\raggedleft\arraybackslash}p{(\linewidth - 16\tabcolsep) * \real{0.1190}}@{}}
\caption{Profils colonnes pour quelques pays et causes
sélectionnés}\tabularnewline
\toprule\noalign{}
\begin{minipage}[b]{\linewidth}\raggedright
\end{minipage} & \begin{minipage}[b]{\linewidth}\raggedleft
C1
\end{minipage} & \begin{minipage}[b]{\linewidth}\raggedleft
C2
\end{minipage} & \begin{minipage}[b]{\linewidth}\raggedleft
C3
\end{minipage} & \begin{minipage}[b]{\linewidth}\raggedleft
C4
\end{minipage} & \begin{minipage}[b]{\linewidth}\raggedleft
C5
\end{minipage} & \begin{minipage}[b]{\linewidth}\raggedleft
C6
\end{minipage} & \begin{minipage}[b]{\linewidth}\raggedleft
C7
\end{minipage} & \begin{minipage}[b]{\linewidth}\raggedleft
C8
\end{minipage} \\
\midrule\noalign{}
\endfirsthead
\toprule\noalign{}
\begin{minipage}[b]{\linewidth}\raggedright
\end{minipage} & \begin{minipage}[b]{\linewidth}\raggedleft
C1
\end{minipage} & \begin{minipage}[b]{\linewidth}\raggedleft
C2
\end{minipage} & \begin{minipage}[b]{\linewidth}\raggedleft
C3
\end{minipage} & \begin{minipage}[b]{\linewidth}\raggedleft
C4
\end{minipage} & \begin{minipage}[b]{\linewidth}\raggedleft
C5
\end{minipage} & \begin{minipage}[b]{\linewidth}\raggedleft
C6
\end{minipage} & \begin{minipage}[b]{\linewidth}\raggedleft
C7
\end{minipage} & \begin{minipage}[b]{\linewidth}\raggedleft
C8
\end{minipage} \\
\midrule\noalign{}
\endhead
\bottomrule\noalign{}
\endlastfoot
AFG & 0.0066205 & 0.0010940 & 0.0015440 & 0.0049481 & 0.0008240 &
0.0071161 & 0.0121089 & 0.0205699 \\
ARM & 0.0000212 & 0.0005381 & 0.0004659 & 0.0000159 & 0.0000000 &
0.0001856 & 0.0003694 & 0.0000408 \\
MLI & 0.0265160 & 0.0007557 & 0.0008961 & 0.0591263 & 0.0279850 &
0.0041887 & 0.0055897 & 0.0179312 \\
FRA & 0.0009573 & 0.0283575 & 0.0220484 & 0.0188297 & 0.0000000 &
0.0039820 & 0.0012990 & 0.0002343 \\
MOZ & 0.0115594 & 0.0009424 & 0.0009402 & 0.0140408 & 0.0319154 &
0.0030202 & 0.0043727 & 0.0105549 \\
AUS & 0.0001948 & 0.0069279 & 0.0066253 & 0.0006523 & 0.0000000 &
0.0008268 & 0.0007775 & 0.0000662 \\
ARG & 0.0019188 & 0.0074241 & 0.0083898 & 0.0052901 & 0.0000000 &
0.0023875 & 0.0069563 & 0.0019205 \\
USA & 0.0048542 & 0.0887060 & 0.0888084 & 0.0242233 & 0.0000000 &
0.0152487 & 0.0427591 & 0.0050431 \\
\end{longtable}

De manière complémentaire, les profils colonnes décrivent, pour chaque
cause de décès, la répartition relative des pays. Chaque valeur indique
la contribution d'un pays donné au total des décès observés pour une
cause spécifique . En 2019, l'Afghanistan représentait 0,66\% (0.0066=
1563/(1563 +13+\ldots+2065+1450))de la mortalité mondiale due à la
méningite. Là encore, la normalisation par colonne permet de comparer
les pays entre eux pour une cause donnée, sans être influencé par les
différences globales de population ou de mortalité.

\subsubsection{\texorpdfstring{Comparaison des pays, graphiques de
profils ligne et colonnes avec
\texttt{ggplot2}}{Comparaison des pays, graphiques de profils ligne et colonnes avec ggplot2}}\label{comparaison-des-pays-graphiques-de-profils-ligne-et-colonnes-avec-ggplot2}

Les profils lignes sont ensuite représentés graphiquement afin de
faciliter l'interprétation. Le profil ligne de la France met en évidence
la prédominance des maladies non infectueuses dans la structure de la
mortalité, confirmant les observations issues de l'analyse par
diagrammes en barres. Une comparaison des profils lignes de la France et
du Mali souligne des contrastes marqués : la France présente une
concentration plus forte des décès sur les maladies chroniques, tandis
que le Mali affiche une part relativement plus importante de maladies
infectueuses et de causes évitables. Ces différences traduisent des
stades distincts de transition épidémiologique et fournissent une base
pertinente pour des analyses multivariées ultérieures, telles que
l'analyse des correspondances.

\pandocbounded{\includegraphics[keepaspectratio]{rapport_files/figure-pdf/unnamed-chunk-6-1.pdf}}

\pandocbounded{\includegraphics[keepaspectratio]{rapport_files/figure-pdf/unnamed-chunk-6-2.pdf}}

\subsection{Analyses avancées ( AFC et
Tests)}\label{analyses-avancuxe9es-afc-et-tests}

\subsubsection{Test du khi-deux et V de
Cramer}\label{test-du-khi-deux-et-v-de-cramer}

Afin d'évaluer statistiquement l'existence d'un lien entre les pays et
les causes de décès, un test du khi-deux d'indépendance est réalisé à
partir du tableau de contingence. L'hypothèse nulle H0 suppose
l'indépendance entre le pays et la cause de décès, tandis que
l'hypothèse alternative H1 postule l'existence d'une dépendance entre
ces deux variables.

Le test met en évidence une statistique du khi-deux très élevée (36 396
969) associée à une p-value nulle, conduisant au rejet de l'hypothèse
d'indépendance au seuil usuel. Cela suggère que la répartition des
causes de décès diffère significativement selon les pays considérés.
Toutefois, l'interprétation de ce résultat doit être nuancée, car la
structure du tableau, notamment la présence de nombreux effectifs élevés
et de fortes disparités entre pays limite la pertinence stricte du test
du khi-deux.

Afin de quantifier l'intensité de la liaison entre les deux variables,
le coefficient de V de Cramér est calculé. La valeur obtenue étant 0.149
indique une liaison faible entre le pays et la cause de décès, malgré le
rejet de l'hypothèse d'indépendance. Ce résultat souligne que la
dépendance statistique observée est réelle mais modérée.

L'examen des effectifs attendus sous l'hypothèse d'indépendance ainsi
que des résidus standardisés permet d'identifier les principales
contributions au khi-deux. Ces écarts traduisent des phénomènes
d'attirance et de répulsion : un résidu standardisé positif indique
qu'un pays et une cause se rencontrent plus fréquemment que prévu
(attirance), tandis qu'un résidu standardisé négatif montre une
occurrence moindre que prévu (répulsion). Un résidu supérieur à 2 ou
inférieur à −2 est généralement considéré comme significatif.

À titre d'illustration, le Bangladesh (BGD) présente un résidu
standardisé de −99,09 pour la malaria (C5), ce qui traduit une répulsion
extrêmement marquée : cette cause est très fortement sous-représentée
dans ce pays par rapport à ce qui serait attendu sous l'hypothèse
d'indépendance. À l'inverse, l'Afghanistan (AFG) affiche un résidu
standardisé très élevé (116,45) pour les troubles maternels (C8),
indiquant une attirance très forte entre ce pays et cette cause de
décès, largement surreprésentée par rapport au modèle d'indépendance.

\subsection{Lancement de l'AFC}\label{lancement-de-lafc}

Pour explorer les relations entre les pays et les causes de décès, une
analyse factorielle des correspondances (AFC) est réalisée. Cette
méthode multivariée est particulièrement adaptée aux tableaux de
contingence et permet de visualiser les dépendances entre lignes (pays)
et colonnes (causes) tout en réduisant la dimensionnalité des données.

\pandocbounded{\includegraphics[keepaspectratio]{rapport_files/figure-pdf/unnamed-chunk-8-1.pdf}}

\pandocbounded{\includegraphics[keepaspectratio]{rapport_files/figure-pdf/unnamed-chunk-8-2.pdf}}

\pandocbounded{\includegraphics[keepaspectratio]{rapport_files/figure-pdf/unnamed-chunk-8-3.pdf}}

L'AFC est appliquée aux 204 pays du tableau, considérés comme lignes
actives. Les valeurs propres associées aux axes factoriels sont
examinées afin de déterminer la part d'inertie expliquée par chaque axe
et d'identifier le nombre d'axes significatifs à retenir. La
décroissance des valeurs propres est représentée graphiquement pour
faciliter l'interprétation. La somme des valeurs propres correspond à
l'inertie totale du tableau, qui peut également être calculée à partir
de la statistique du khi-deux normalisée par l'effectif total (Phi2).
Dans les deux cas la valeur trouvée est 0.6695183.

Le critère de Kaiser est utilisé pour identifier les axes factoriels
dont l'inertie dépasse la moyenne qui est 0.02231728, indiquant qu'ils
expliquent une proportion d'information supérieure à celle attendue par
hasard. Dans cette analyse, six axes présentent une inertie supérieure à
la moyenne et sont donc retenus pour l'interprétation.

\pandocbounded{\includegraphics[keepaspectratio]{rapport_files/figure-pdf/unnamed-chunk-9-1.pdf}}

\pandocbounded{\includegraphics[keepaspectratio]{rapport_files/figure-pdf/unnamed-chunk-9-2.pdf}}

Les diagrammes de l'inertie totale et du pourcentage d'inertie
permettent de visualiser la contribution relative de chaque axe à la
structure globale des données, préparant ainsi l'interprétation des
positions des pays et des causes sur les plans factoriels. Cette étape
constitue un prérequis essentiel avant de produire les représentations
graphiques des individus et des variables dans l'espace factoriel.

\subsubsection{Critère du bâton brisé pour sélectionner les
axes}\label{crituxe8re-du-buxe2ton-brisuxe9-pour-suxe9lectionner-les-axes}

Pour compléter l'identification des axes significatifs, le critère du
bâton brisé est utilisé. Les valeurs propres observées sont ainsi mises
en relation avec les proportions théoriques du bâton brisé, permettant
de distinguer les axes apportant une information réelle de ceux qui ne
contiennent qu'un bruit statistique.

\begin{verbatim}
 dim 1  dim 2  dim 3  dim 4  dim 5  dim 6  dim 7  dim 8  dim 9 dim 10 dim 11 
  TRUE   TRUE   TRUE   TRUE  FALSE  FALSE  FALSE  FALSE  FALSE  FALSE  FALSE 
dim 12 dim 13 dim 14 dim 15 dim 16 dim 17 dim 18 dim 19 dim 20 dim 21 dim 22 
 FALSE  FALSE  FALSE  FALSE  FALSE  FALSE  FALSE  FALSE  FALSE  FALSE  FALSE 
dim 23 dim 24 dim 25 dim 26 dim 27 dim 28 dim 29 dim 30 
 FALSE  FALSE  FALSE  FALSE  FALSE  FALSE  FALSE  FALSE 
\end{verbatim}

\pandocbounded{\includegraphics[keepaspectratio]{rapport_files/figure-pdf/unnamed-chunk-10-1.pdf}}

La comparaison numérique indique que les axes dont l'inertie observée
dépasse celle du bâton brisé sont considérés comme pertinents. Dans
cette analyse, quatre axes présentent une inertie supérieure à la valeur
attendue par le bâton brisé et sont donc retenus pour l'interprétation
factorielle.

Un diagramme comparatif illustre visuellement la sélection des axes :
les barres rouges représentant l'inertie observée sont mises côte à côte
avec celles du bâton brisé, représentées en bleu. Cette représentation
permet de visualiser aisément quels axes dépassent le seuil et
confirment leur pertinence pour l'étude des relations entre les pays et
les causes de décès. Ce critère identifiant moins d'axes pertinents que
le critère de Kaiser, son résultat sera privilégié pour la suite de
l'analyse.

\subsubsection{Analyse des contributions des lignes et
colonnes}\label{analyse-des-contributions-des-lignes-et-colonnes}

Après sélection des axes significatifs, l'AFC permet de représenter les
modalités des lignes (pays) et des colonnes (causes) dans l'espace
factoriel. Les diagrammes factoriels des lignes et des colonnes mettent
en évidence les similitudes et les différences structurelles : les pays
proches sur le plan factoriel présentent des profils de mortalité
similaires, comme Djibouti et le Congo, tandis que les causes de décès
rapprochées indiquent une co-occurrence relative dans différents pays.
Par exemple, les troubles néonatals (C14), troubles maternels (C8) et
maladies diarrhéiques(C18) apparaissent proches sur le graphique,
suggérant qu'elles sont présentes dans des proportions comparables dans
les mêmes pays. À l'inverse, la malaria (C5), très éloignée de l'origine
et des autres modalités, présente un profil fortement spécifique,
caractéristique d'un nombre limité de pays.

\pandocbounded{\includegraphics[keepaspectratio]{rapport_files/figure-pdf/unnamed-chunk-11-1.pdf}}

\pandocbounded{\includegraphics[keepaspectratio]{rapport_files/figure-pdf/unnamed-chunk-11-2.pdf}}

\pandocbounded{\includegraphics[keepaspectratio]{rapport_files/figure-pdf/unnamed-chunk-11-3.pdf}}

Les biplots combinant simultanément les modalités des lignes (pays) et
des colonnes (causes) permettent de visualiser les relations entre pays
et causes de décès sur différents plans factoriels. Le premier plan
factoriel, défini par les axes 1 et 2, est le plus informatif,
concentrant une part importante de l'inertie totale avec 43 \% pour
l'axe 1 et 13,2 \% pour l'axe 2. Il permet d'identifier les groupes de
pays aux profils de mortalité similaires ainsi que les causes de décès
les plus structurantes.

Les plans factoriels secondaires, explorés grâce aux combinaisons d'axes
1-3, 1-4, 3-2, 4-2 et 4-3, apportent des informations complémentaires
sur les dimensions moins dominantes. Par exemple, le plan 1-3 explique
43 \% et 10,5 \% de l'inertie, le plan 1-4 contribue à 43 \% et 8,7 \%,
et les autres plans permettent de mettre en évidence des variations
supplémentaires entre pays et causes de décès. L'option repel = T
utilisée dans les biplots améliore la lisibilité en évitant le
chevauchement des étiquettes. L'analyse combinée de ces différents plans
factoriels permet ainsi de mettre en évidence à la fois les structures
principales et les relations secondaires dans les données.

\pandocbounded{\includegraphics[keepaspectratio]{rapport_files/figure-pdf/unnamed-chunk-12-1.pdf}}

\pandocbounded{\includegraphics[keepaspectratio]{rapport_files/figure-pdf/unnamed-chunk-12-2.pdf}}

\pandocbounded{\includegraphics[keepaspectratio]{rapport_files/figure-pdf/unnamed-chunk-12-3.pdf}}

\pandocbounded{\includegraphics[keepaspectratio]{rapport_files/figure-pdf/unnamed-chunk-12-4.pdf}}

\pandocbounded{\includegraphics[keepaspectratio]{rapport_files/figure-pdf/unnamed-chunk-12-5.pdf}}

Pour approfondir l'interprétation, les contributions des lignes et des
colonnes aux axes factoriels sont examinées. Les modalités les plus
contributives sont identifiées, ce qui permet de se concentrer sur les
pays et causes qui expliquent le mieux la variance sur chaque axe. Étant
donné le grand nombre de pays (204), une sélection des dix premières
contributions pour chaque axe facilite la lecture et l'interprétation
des résultats. Sur l'axe 1, le Nigeria (NGA, ≈20\%), la Chine et le
Mozambique sont les principaux contributeurs, tandis que la malaria (C5)
domine parmi les causes de décès (≈21\%). L'axe 2 est fortement
influencé par l'Afrique du Sud (ZAF, 33,8\%) et le VIH (C9) (≈68\%).
L'axe 3 met en avant l'Afghanistan (AFG, 50,9\%) et les conflits et
terrorisme C21 (plus de 80\%), et l'axe 4 est porté par l'Inde (IND,
24,2\%) et la malaria (plus de 40\%).

\pandocbounded{\includegraphics[keepaspectratio]{rapport_files/figure-pdf/unnamed-chunk-13-1.pdf}}

\pandocbounded{\includegraphics[keepaspectratio]{rapport_files/figure-pdf/unnamed-chunk-13-2.pdf}}

\pandocbounded{\includegraphics[keepaspectratio]{rapport_files/figure-pdf/unnamed-chunk-13-3.pdf}}

\pandocbounded{\includegraphics[keepaspectratio]{rapport_files/figure-pdf/unnamed-chunk-13-4.pdf}}

\pandocbounded{\includegraphics[keepaspectratio]{rapport_files/figure-pdf/unnamed-chunk-13-5.pdf}}

\pandocbounded{\includegraphics[keepaspectratio]{rapport_files/figure-pdf/unnamed-chunk-13-6.pdf}}

\pandocbounded{\includegraphics[keepaspectratio]{rapport_files/figure-pdf/unnamed-chunk-13-7.pdf}}

\pandocbounded{\includegraphics[keepaspectratio]{rapport_files/figure-pdf/unnamed-chunk-13-8.pdf}}

La qualité de représentation de chaque modalité est évaluée à l'aide du
cos², qui mesure la proportion de la variance de la modalité expliquée
par les axes retenus. Les modalités présentant un cos² élevé sont mieux
représentées sur le plan factoriel et constituent des points de
référence fiables pour l'interprétation. Pour rendre les graphiques plus
lisibles, nous avons d'abord filtré les modalités avec un cos² supérieur
à 0,7, garantissant ainsi que seules celles bien expliquées par les axes
sont affichées. Parmi ces modalités, les 20 mieux représentées ont été
sélectionnées pour produire un graphique final. Ce dernier utilise un
code couleur : les modalités les mieux représentées, avec un cos² proche
de 1, apparaissent en couleur claire, tandis que celles avec un cos²
proche de 0,7 apparaissent en couleur plus foncée.

\pandocbounded{\includegraphics[keepaspectratio]{rapport_files/figure-pdf/unnamed-chunk-14-1.pdf}}

\pandocbounded{\includegraphics[keepaspectratio]{rapport_files/figure-pdf/unnamed-chunk-14-2.pdf}}

\pandocbounded{\includegraphics[keepaspectratio]{rapport_files/figure-pdf/unnamed-chunk-14-3.pdf}}

\pandocbounded{\includegraphics[keepaspectratio]{rapport_files/figure-pdf/unnamed-chunk-14-4.pdf}}

\pandocbounded{\includegraphics[keepaspectratio]{rapport_files/figure-pdf/unnamed-chunk-14-5.pdf}}

Cette analyse graphique et quantitative des contributions et de la
qualité de représentation fournit une base solide pour tirer des
conclusions sur les profils de mortalité et les relations entre pays et
causes dans l'ensemble du jeu de données.

\subsubsection{Distances au centre de
gravité}\label{distances-au-centre-de-gravituxe9}

L'AFC permet de mesurer l'écart de chaque pays ou cause par rapport au
profil moyen à travers la distance au centre de gravité (barycentre).
Cette distance est calculée en rapportant l'inertie de chaque modalité à
sa masse. Par exemple, dans notre analyse, le Bangladesh (BGD) présente
la distance la plus faible (≈0,12), ce qui signifie que son profil est
très proche du profil moyen, tandis que l'Afghanistan (AFG) a la
distance la plus élevée (≈11,02), indiquant un profil fortement
atypique.

Pour les causes, certaines présentent également des écarts importants :
la cause C21 se distingue par une distance très élevée (≈60,83), ce qui
suggère un profil extrêmement spécifique par rapport aux autres causes.
Ces mesures permettent ainsi de détecter les pays et causes qui se
rapprochent le plus du profil moyen et ceux qui s'en écartent fortement.

La visualisation de ces distances, sous forme de diagrammes en barres
verticales ou horizontales, facilite l'interprétation. Les barres les
plus courtes correspondent aux profils proches de la moyenne, tandis que
les barres plus longues signalent des profils atypiques. Cette
représentation claire guide l'interprétation des plans factoriels
précédemment étudiés et aide à identifier les modalités les plus
influentes ou singulières dans l'analyse.

\pandocbounded{\includegraphics[keepaspectratio]{rapport_files/figure-pdf/unnamed-chunk-15-1.pdf}}

\pandocbounded{\includegraphics[keepaspectratio]{rapport_files/figure-pdf/unnamed-chunk-15-2.pdf}}

\pandocbounded{\includegraphics[keepaspectratio]{rapport_files/figure-pdf/unnamed-chunk-15-3.pdf}}

\pandocbounded{\includegraphics[keepaspectratio]{rapport_files/figure-pdf/unnamed-chunk-15-4.pdf}}

\section{Conclusion}\label{conclusion}

\textbf{Fatoumata}

L'objectif de ce travail était d'étudier l'existence d'un lien entre les
causes de décès et les pays, et d'identifier d'éventuelles similarités
de profils de mortalité entre ces derniers, en mobilisant l'analyse
factorielle des correspondances. Les résultats obtenus permettent
d'apporter des réponses claires aux questions formulées en introduction.

Tout d'abord, le test du khi-deux, complété par l'indice de V de Cramer,
met en évidence l'existence d'une dépendance entre les pays et les
causes de décès. Bien que cette dépendance demeure d'intensité modérée,
elle est néanmoins significative et confirmée par la structure révélée
par l'AFC. Cette analyse montre que les causes de décès ne se
répartissent pas aléatoirement selon les pays, mais qu'elles sont
associées à des profils nationaux spécifiques.

L'examen du plan factoriel met en lumière des regroupements cohérents de
pays présentant des profils de mortalité similaires. Les pays dits
développés, majoritairement situés en Europe, en Amérique du Nord ou
dans certaines régions d'Asie, apparaissent proches du profil moyen.
Leur mortalité est principalement dominée par des maladies chroniques et
dégénératives, telles que les maladies cardiovasculaires, les maladies
neurologiques ou celles liées au vieillissement. À l'inverse, de
nombreux pays africains se situent à distance du centre de gravité et
présentent des profils proches les uns des autres, caractérisés par une
forte prévalence de maladies infectieuses, de la malnutrition, ainsi que
de décès liés aux conflits ou à l'instabilité politique. Ces causes,
fortement contributrices aux axes factoriels, expliquent une part
importante de la dépendance observée. De manière similaire, certains
pays en situation de crise hors du continent africain, tels que
l'Afghanistan ou le Yémen, se distinguent également par des profils très
éloignés du profil moyen, en lien avec des causes de mortalité
spécifiques comme les conflits armés, le terrorisme, le paludisme ou le
VIH.

Ainsi, les résultats obtenus apparaissent globalement cohérents avec le
contexte géographique, géopolitique et socio-économique mondial. Ils
confirment l'existence d'un lien entre le niveau de développement des
pays et la structure de leur mortalité, ainsi que la présence de profils
de décès similaires parmi des pays partageant des conditions de
développement et de stabilité comparables. Ces observations doivent
toutefois être interprétées avec prudence, au regard des limites liées à
l'année étudiée et à l'agrégation des causes de décès.

\textbf{Yester}

L'analyse des données de mortalité par pays et par cause a permis
d'illustrer la richesse et la diversité des profils de décès dans le
monde. L'exploration initiale par diagrammes en barres a montré des
différences marquées entre pays, notamment entre pays à revenu élevé et
pays à revenu faible. Les profils lignes et colonnes ont permis de
comparer les structures de mortalité de manière normalisée et
d'identifier les causes et pays les plus contributifs à chaque axe
factoriel.

Le test du khi-deux a confirmé l'existence d'une dépendance statistique
entre pays et causes de décès, bien que la force de cette liaison soit
modérée. L'AFC a permis de synthétiser cette information dans un espace
de faible dimension, en sélectionnant les axes les plus pertinents selon
les critères de Kaiser et du bâton brisé. L'examen des plans factoriels,
des contributions et des distances au centre de gravité a fourni une
compréhension détaillée des profils atypiques et des similarités entre
pays et causes.

En conclusion, cette approche méthodologique a permis d'identifier les
tendances globales et les particularités locales de la mortalité,
fournissant une base solide pour des analyses comparatives et pour des
prises de décision en santé publique. Les résultats mettent en évidence
à la fois les déterminants communs et les spécificités nationales des
causes de décès, illustrant l'intérêt des méthodes multivariées pour
l'analyse de données de grande dimension.

\section{Annexe}\label{annexe}

\subsection{Source du jeu de données
initiale}\label{source-du-jeu-de-donnuxe9es-initiale}

URL :
https://www.kaggle.com/datasets/iamsouravbanerjee/cause-of-deaths-around-the-world

\subsection{Tableau de l'encodage dzs différentes causes de
décès}\label{tableau-de-lencodage-dzs-diffuxe9rentes-causes-de-duxe9cuxe8s}

\begin{longtable}[]{@{}ll@{}}
\caption{Encodage des causes de décès}\tabularnewline
\toprule\noalign{}
Meningitis & C1 \\
\midrule\noalign{}
\endfirsthead
\toprule\noalign{}
Meningitis & C1 \\
\midrule\noalign{}
\endhead
\bottomrule\noalign{}
\endlastfoot
Alzheimer's Disease and Other Dementias & C2 \\
Parkinson's Disease & C3 \\
Nutritional Deficiencies & C4 \\
Malaria & C5 \\
Drowning & C6 \\
Interpersonal Violence & C7 \\
Maternal Disorders & C8 \\
HIV/AIDS & C9 \\
Drug Use Disorders & C10 \\
Tuberculosis & C11 \\
Cardiovascular Diseases & C12 \\
Lower Respiratory Infections & C13 \\
Neonatal Disorders & C14 \\
Alcohol Use Disorders & C15 \\
Self-harm & C16 \\
Exposure to Forces of Nature & C17 \\
Diarrheal Diseases & C18 \\
Environmental Heat and Cold Exposure & C19 \\
Neoplasms & C20 \\
Conflict and Terrorism & C21 \\
Diabetes Mellitus & C22 \\
Chronic Kidney Disease & C23 \\
Poisonings & C24 \\
Protein-Energy Malnutrition & C25 \\
Road Injuries & C26 \\
Chronic Respiratory Diseases & C27 \\
Cirrhosis and Other Chronic Liver Diseases & C28 \\
Digestive Diseases & C29 \\
Fire, Heat, and Hot Substances & C30 \\
Acute Hepatitis & C31 \\
\end{longtable}




\end{document}
